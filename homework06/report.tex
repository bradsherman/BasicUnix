% Report on the demographics in CSE at Notre Dame


\documentclass{article}

\title{{\bf Demographics in Computer Science at Notre Dame}}
\author{{\it Brad Sherman} \\ {\it bsherma1@nd.edu} }

\usepackage{graphicx}
\usepackage{booktabs}
\usepackage[margin=1in]{geometry}
\usepackage{hyperref}

\begin{document}

\maketitle

\section*{Overview}

For this report I first had to make a script to handle the demographic
data. My script, {\bf demo\_stats.sh} takes the data and counts how many
male, female, caucasian, oriental, hispanic, african american, native
american, multiple ethnicities, or undeclared students there are for each 
graduating class starting with the class of 2013. It then prints these 
numbers to a file called demog.dat which is used to create Figure 
\ref{fig:Demographics}, Table \ref{tab:Table1}, and Table \ref{tab:Table2}.
\\
My main takeaway from this exercise is that the Notre Dame CSE undergraduate
pool is mainly caucasian males. I expected it to be mostly caucasian 
the whole Notre Dame student body is mostly caucasian, but I was a little
surprised at how many more males there were than females. There are at 
least twice as many males as females for each year, sometimes even triple
or double. 

\section*{Methodology}

In order to process the data, I had to make 10 associative arrays in my 
script. One for the years, two for males/females, and 7 for the different
ethnicities listed in the file. I used awk with a comma field separator
with the search pattern "//" which grabs everything, and a series of if
statements checks to see which letter it is, and increments the corresponding
associated array. At the end of the script the value of each associated
array is printed on the same row as its corresponding year into demog.dat.
\\
For Figure \ref{fig:Demographics} I used gnuplot with a cluster histogram style to show the
number of students belonging to each demographic for each year. The most
frustrating part of this process was trying to get the labels on the x-axis
to line up with the clusters. I overcame it by doing much research on gnuplot
and getting help from Professor Bui. 

\section*{Analysis}

Table \ref{tab:Table1} shows a glaring difference in the number of males vs.
females in the computer science department at Notre Dame.
\\
\begin{table}[h!]
\centering
\begin{tabular}{c|c|c}
Year & Male & Female\\
\hline
2013&49&14\\
2014&44&12\\
2015&58&16\\
2016&60&19\\
2017&65&26\\
2018&90&36\\
\end{tabular}
\caption{The number of Male vs. Female in past years in CSE at Notre Dame.}
\label{tab:Table1}
\end{table}
\\
Table \ref{tab:Table2} shows another disparity, this time between caucasian and all
other ethnicities in the department.
\\
\begin{table}[h!]
\centering
\begin{tabular}{c|c|c|c|c|c|c|c}
Year & Caucasian & Oriental & Hispanic & African American & Native American & Multiple & Undeclared\\
\hline
2013&43&7&7&3&1&2&0\\
2014&43&5&4&2&1&1&0\\
2015&47&9&10&4&1&1&2\\
2016&53&9&9&1&7&0&0\\
2017&60&12&3&5&5&6&0\\
2018&91&8&12&3&4&8&0\\
\end{tabular}
\caption{The number of students belonging to different ethnicities.}
\label{tab:Table2}
\end{table}
\\
Below, in Figure \ref{fig:Demographics}, one can clearly see just how dominated 
the CSE department at Notre Dame is by caucasian males.
\\
\begin{figure}[h!]
\includegraphics[width=\linewidth]{demographics.png}
\caption{A histogram showing the number of computer science students belonging to different demographics.}
\label{fig:Demographics}
\end{figure}
\\
\section*{Discussion}

I think the issues of gender and ethnic diversity are very important. I
believe that all people should have an equal chance to pursue whatever 
career they want, given that they are qualified. I think the department
should work to increase diversity, but that they should also look at these 
numbers with a grain of salt. Notre Dame as a whole is a predominantly 
caucasian campus, along with slightly more males than females. I think at
least for ethnic diversity, the department should strive to be at or above
the percentage of each ethnicity throughout the whole University. However
for gender diversity, the department should encourage more females to join 
computer science (as they have been) so the numbers are a little more equal.
\\

\end{document}


